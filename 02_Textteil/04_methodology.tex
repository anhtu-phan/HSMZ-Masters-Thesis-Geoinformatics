\chapter{Methodology}
\label{chapter:methodology}

This feasibility study is based on three essential parts. The first is a reference implementation, providing insights on the work necessary to arrive at the base functionality. After the implementation, a quantitative analysis of the application's functionality is made, as well as a statistical overview of the time spent on development. Additionally, there is a qualitative review of the resulting codebase and a reflection on the work process.

\section{Reference Implementation}

\subsection{Choice of concepts and tools}

To produce a valid test subject for the proposal, the reference implementation is created according to a prior selection of tools and methods deemed appropriate for the task at hand. The choice is made from the range presented in \autoref{chapter:concepts} and \autoref{chapter:tools}.

\subsection{Application development and deployment}

The application is implemented in its entirety, documented and packaged. Appropriate test coverage is provided, and the overall time spent is logged in timesheets and categorised by the general work areas. The application's server components are deployed to university hardware and made available over the internet. The client application is then run on various consumer computer systems.

\section{Quantitative Analysis}

\subsection{Performance Testing}

The application's performance is only tested regarding the load put on the \ac{CPU} (server and client) as well as the network throughput and latency. It is verified that all signal processing works as expected through unit testing and simple testing tasks performed on the application. A practical test using actual performers and dance interaction is beyond the scope of this feasibility study.

\subsection{Time spent}

A statistical analysis of the timesheets provides insight into the time spent on various aspects of the software. It should differentiate between basic boilerplate code that can be reused and custom code used for the actual use case.

\section{Qualitative Analysis}

\subsection{Code Quality}

The code quality is mainly assessed in terms of volume (lines of code), complexity (number of classes and functions, cognitive complexity) and stylistic coherence.

\subsection{Critical reflection}

A critical analysis of the development process should weigh the expectations against the experiences made during the implementation of the decisions made in planning the application. It should critically evaluate the feasibility and discuss the benefits and drawbacks of establishing a task-specific application from scratch.

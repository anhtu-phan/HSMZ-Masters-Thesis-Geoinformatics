\chapter{Concepts}

\section{Telepresence}

\section{Motion Capture}

\section{Movement Sonification}

\section{Application Design Paradigms}

\subsection{Single Page Applications}

The concept of a \ac{SPA} originated around the beginning of the 2000s with the terms "Inner-Browsing" \parencite{innerBrowsing} and \ac{AJAX} \parencite{ajaxNewApproach}. It breaks with the traditional way of moving from one page to another in favour of asynchronous loading and replacing parts of the current page. This allows for a website to evoke the look and feel of a regular desktop application.

\subsection{Progressive Web Applications}

The term \ac{PWA} was initially coined in 2015 by two Google employees in an online Article \parencite{progressiveWebApplications}. At its core, it describes the process of a website "progressively" evolving into a true device application by adding offline functionality and blending with the operating system functionality. It is often built atop the concept of an \ac{SPA} and can be perceived by the user as an application they own instead of just accessed at a remote location.

\subsection{Real-time Web Applications}

A real-time web application enhances the user experience by relaying relevant changes on the server to the server as they happen. This can be a simple chat application or a more complex collaborative multi-user environment. While real-time updates can happen on any multi-page website, they can also be a beneficial feature of an \ac{SPA} or a \ac{PWA}. Instantaneous updates are commonly realised using WebSockets, a transmission protocol that was standardised as \ac{RFC} 6455 by the \ac{IETF} in 2011 \parencite{webSocketsProtocolRfc}. It allows full-duplex communication between client and server, running on the same ports and in the same transport layer as the half-duplex \ac{HTTP} protocol, thus being compatible with existing web infrastructure. It allows for updates to be pushed to the client whenever a resource on the server changes.



\section{Application Deployment}

\subsection{Containerisation}

\subsection{Orchestration}


\pdfbookmark{Abstract}{abstract}
\addchap*{Abstract}

This study investigates the feasibility of creating a fully customised and task-specific telepresence application based exclusively on open web standards and free software for use in contemporary dance practice.
It surveys existing technologies and paradigms and establishes a practical reference implementation to evaluate its basic functionality and the development process.
The study concludes with a positive assessment of the existing technological landscape and the feasibility to produce task-specific web applications as an intrinsic component in a smaller interdisciplinary projects with a strong focus on digital practice.

Keywords: open-source software, telepresence, contemporary dance, motion capture
 
 Die vorliegende Studie untersucht die Machbarkeit der Entwicklung einer aufgabenspezifischen Telepräsenzanwendung, ausschließlich auf offenen Webstandards und freier Software basierend, für den Einsatz in der zeitgenössischen Tanzpraxis.
 Es wird ein Überblick über existente Technologien und Paradigmen erabeitet und eine Referenzimplementierung erstellt, deren grundlegende Funktionalität evaluiert und der Prozess ihrer Entwicklung bewertet wird.
 Die Studie schließt mit einer positiven Beurteilung, sowohl im Hinblick auf die generelle technologische Landschaft, als auch die Machbarkeit einer aufgabenspezifischen Softwareimplementierung als intrinsischer Komponente von kleineren interdisziplinären Projekten mit einem starken Fokus auf digitaler Praxis.
 
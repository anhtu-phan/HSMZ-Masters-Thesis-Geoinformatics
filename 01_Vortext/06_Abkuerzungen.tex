% Erklärender Text zu dieser Datei --------------------------------------------------------
% Die Datei 06_Abkuerzungen.tex dient der Definition von Abkürzungen, die im Abkürzungsverzeichnis gelistet werden sollen.
% Zur besseren Übersichtlichkeit wird empfohlen die Abkürzungen zentral in dieser Datei zu definieren.
%
% Abkürzungen müssen erst definiert werden mittels \acro bevor diese im eigentlichen Textteil mit \ac verwendet werden können.
% Die Angabe von Übersetzungen zu Abkürzungen ist möglich mittels der Anweisung \acroforeign.
% -----------------------------------------------------------------------------------------
\begin{acronym}
	\acro{3D}{3-dimensional}
	\acro{AJAX}{Asynchronous JavaScript and XML}
	\acro{API}{Application Programming Interface}
	\acro{AR}{Augmented Reality}
	\acro{CLI}{Command-line interface}
	\acro{CNCF}{Cloud Native Computing Foundation}
	\acro{CSS}{Cascading Style Sheets}
	\acro{ES6}{ECMAScript 6}
	\acro{HTML}{HyperText Markup Language}
	\acro{HTTP}{Hypertext Transmission Protocol}
	\acro{IETF}{Internet Engineering Task Force}
	\acro{JSX}{JavaScript XML}
	\acro{JS}{JavaScript}
	\acro{NPM}{Node Package Manager}
	\acro{OCI}{Open Container Initiative}
	\acro{PWA}{Progressive Web Application}
	\acro{RFC}{Request For Comments}
	\acro{RTC}{Real-time Communication}
	\acro{SPA}{Single Page Application}
	\acro{TS}{TypeScript}
	\acro{UI}{User Interface}
	\acro{VR}{Virtual Reality}
	\acro{W3C}{World Wide Web Consortium}
	\acro{WebRTC}{Web Real-Time Communication}
	\acro{WebXR}{Web Mixed Reality}
	\acro{XML}{Extensible Markup Language}
	\acro{XR}{Mixed Reality}
\end{acronym}
